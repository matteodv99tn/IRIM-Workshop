\documentclass{article}
\usepackage{graphicx} % Required for inserting images
\usepackage[width=15cm, height=23cm]{geometry}
\usepackage{hyperref}
\usepackage{changepage}

\title{IRIM3D 2025 Workshop Proposal}
\author{matteo.dallevedove }
\date{July 2025}

\newcommand{\email}[1]{\href{mailto:#1}{\itshape #1}}

\begin{document}

% \maketitle
\begin{center}
    {\Large IRIM3D 2025 Workshop Proposal:} \vspace{5mm} \\
    {\LARGE  Industry 5.0: Workplace Transformation \\ with Next Generation Smart Robots}
\end{center}

\noindent
\href{https://i-rim.it/wp-content/uploads/2025/06/Call-for-workshops-I-RIM-3D-2025.pdf}{Submission Guidelines Link (to be removed)}

\begin{quote}
    % \paragraph{\itshape Abstract}
    % \textbf{Abstract:} 
    Industry 5.0 brings humans back to the center of industrial production through effective human-robot collaboration.
    While collaborative robots enable shared workspaces, true integration requires more than safety certification. 
    This workshop explores practical challenges and recent advances in perception, control, and motion planning to bridge the gap between lab research and real-world deployment.
\end{quote}

% \paragraph{\itshape Abstract}
% Industry 5.0 is about restoring the pivotal role of the human in the industrial workplace.
% With the raise of collaborative robotics, we see lots of robotics cell which can work alongside the human, making the shopfloor a fluid space for human-machine physical interaction.
% Still, a safety certification is not enough to establish a safe and useful human-robot collaboration that makes both the human more involved in the production process, and the process itself more efficient.
% This workshop will explore latest trends in research that can effectively enable Industry 5.0 in a practical context, addressing all problems rising when transitioning from the laboratory to a real-world application. 
% The workshop will explore a diverse set of technology ranging from advanced perception, to low-level control strategies for a safe human-robot interaction and human-aware motion planning.
% 
% Industry 5.0 places the human back at the center of the industrial workplace, fostering a new era where people and machines collaborate seamlessly. With the rise of collaborative robotics, shop floors are evolving into dynamic environments where humans and robots share physical space and tasks. However, achieving a truly effective and safe human-robot collaboration goes far beyond obtaining safety certifications. It requires rethinking processes to enhance both human involvement and overall production efficiency.
% 
% This workshop will delve into the latest research trends that are making Industry 5.0 a practical reality. We will tackle the challenges that arise when moving from controlled laboratory settings to real-world industrial applications. Topics will span a wide array of technologies, including advanced perception systems, low-level control strategies for safe human-robot interaction, and human-aware motion planning. By bringing together these diverse perspectives, the workshop aims to highlight solutions that not only ensure safety but also enrich human participation and productivity on the factory floor.

\paragraph*{1. Workshop Title} 
Industry 5.0: Workplace Transformation with Next Generation Smart Robots
% \begin{itemize}
%     \item Pathways to Industrial Impact: Flexible Manufacturing from Research to Reality;
%     \item Industry 5.0: Workplace Transformation from Concept to Practice
%     \item Shopfloor Ready: Turning Flexible Manufacturing Concepts into Capabilities
% \end{itemize}
\subsubsection*{2. Organisers}
\begin{itemize}
    \item Matteo Dalle Vedove, Università degli Studi di Trento, \email{matteo.dallevedove@unitn.it};
    \item Edoardo Lamon, Università degli Studi di Trento, \email{edoardo.lamon@unitn.it};
    \item Daniele Fontanelli, Università degli Studi di Trento, \email{daniele.fontanelli@unitn.it};
\end{itemize}

\subsubsection*{3. Workshop Format}
% The workshop will consists of 4-5 keynote talks, each of 15 minutes, given by academic and industry leaders in the context of flexible manufacturing, each one showcasing key technology enablers for a future generation of production lines. 
% 
% Following the keynotes, the invited experts will form a panel discussing the open challenges in bringing these technology to a production-ready level, and the competencies required to integrate different concepts into a unique and cohesive machine that can improve the production line.

The tentative schedule of the workshop is:
\begin{enumerate}
    \item Introduction (5 minutes): overview of workshop themes, illustrated with practical examples from the EU-funded \href{https://www.magician-project.eu/}{Magician} and \href{https://www.inverse-project.org/}{Inverse} projects;
    \item 6 Keynote Talks (14 minutes each): each keynote will address a key-enabling technology for Industry 5.0, or how those concepts can be transferred to the state-of-the-art in the automation industry. Each talk comprises a 12 minute presentation and 2 minutes for Q\&A;
\end{enumerate}


\subsubsection*{4. Workshop Objectives and Main Topics}
The workshop aims at disseminating recent research advancements on smart robotic solutions in Industry 5.0. Rather than being vertical on a specific topic, the workshop will present a broad and diverse selection of research areas, with strong focus on the practical implementation to bring academia closer to the industry needs. 
Topic of interest include:
\begin{itemize}
    % \item stability and passivity for safe robot control in physical human-robot interaction;
    % \item human-aware motion prediction and planning in collaborative and cooperative environments;
    % \item shared autonomy and adaptive control to balance human expertise and robotic consistency;
    % \item learning from demonstration to enhance productivity and flexibility;
    % \item advanced sensing systems and algorithms for smart factories;
    % \item reinforcement learning and large behavioural models for next-generation task allocation and planning;
    % PROPOSTA EDO
    \item advanced sensing systems and algorithms for smart factories;
    \item safe robot control in physical human-robot interaction;
    \item shared autonomy and adaptive control to balance human expertise and robotic consistency;
    \item human-aware motion prediction and planning in collaborative and cooperative environments;
    \item learning from human demonstration to enhance productivity and flexibility;
    \item reinforcement learning and large behavioural models for industrial settings;
    \item next-generation task allocation and planning in human-robot teams.
\end{itemize}


\paragraph{5. Keywords} 
Industrial Robotics, Industry 5.0, Human-Robot Collaboration, Safe Control, Learning from Demonstration.


\paragraph{6. Preferred Date} \textbf{TO BE DECIDED BASED ON KEYNOTE GUEST AVAILABILITY}


\subsubsection*{7. List of Speakers}
Possible list of speaker (not in order):
\begin{itemize}
    \item Gianluca Lentini, \href{https://exsensia.ai/}{Exsensia} (not confirmed) -- Backup:
    Francesca Negrello, Istituto Italiano di Tecnologia (not confirmed);
    \item Antonio Frisoli, \href{https://nextgen-robotics.it/}{Next Generation Robotics} (not confirmed);
    \item Alessio Del Bue, Istituto Italiano di Tecnologia (not confirmed) -- Backup: Daniele Fontanelli;
    \item Riccardo Caccavale, Università degli Studi di Napoli "Federico II" (not confirmed);
    \item Marta Lorenzini, Istituto Italiano di Tecnologia (not confirmed);
    \item Marco Todescato, Fraunhofer Italia (not confirmed);
\end{itemize}
Other names
\begin{itemize}
    \item Arash Ajoudani, Istituto Italiano di Tecnologia (not confirmed);
    \item Cristian Secchi, Università degli Studi di Modena e Reggio Emilia (not confirmed);
    \item Manuel Catalano, Istituto Italiano di Tecnologia (not confirmed);
    \item Paolo Rocco, Politecnico di Milano (not confirmed);
    \item Silvio Traversaro, Istituto Italiano di Tecnologia (not confirmed);
\end{itemize}


\paragraph{8. Expected Number of attendees} 20-30 people.


\paragraph{9. Target Audience} 
% The aim of the workshop is not to be vertical on one topic, but to embrace different fields of research. 
% The workshop targets everyone with interest in smart manufacturing who is curious in uncovering latest trend in this sector.

The workshop targets a broad audience interested in smart manufacturing and Industry 5.0, including:
\begin{itemize}
    \item robotics and control researchers looking to understand industrial requirements;
    \item industry professionals seeking insight into upcoming technologies;
    \item students and early-career scientists eager to explore interdisciplinary applications.
\end{itemize}



\paragraph{10. Workshop Outcomes}
All accepted manuscripts and accompanying posters, upon confirmation from the authors, will be published in the workshop webpage to further disseminate the works.


\paragraph{11. Plan for Special Issue or Session}
% Currently, we do not foresee any publication of special issue on the workshop theme.
% Based on the comments and outcomes of the workshop, we may think of organising another workshop at larger venues, such as ICRA or IROS.
No special issue is currently planned. 
Based on workshop feedback and interest, we may explore organizing a larger session or special track at larger venues.


\paragraph{12. Call for Extended Abstracts} 
We will organise a call for abstracts that will accept up to 10 manuscripts addressing the workshop theme, using the same template and rules of the main event.
All accepted works will be reviewed by the organisation committee, and will have the possibility to disseminate their work through a poster presentation.
The top 3 submission will also be eligible for a 5 minute oral presentation.
% Upon review of the organisation committee, we will give to the 3 best paper the possibility to showcase their work with an oral presentation.


\paragraph{13. Workshop Webpage} \textbf{TODO}


\paragraph{14. Notes on Registration}
The 2 free one-day registrations will be assigned using the following policy:
\begin{enumerate}
    \item invited speakers who do not register for the main IRIM3D event;
    \item organizing committee members not otherwise registered; 
    \item authors of accepted abstracts who are not registered for IRIM3D.
    % \item to invited speakers who decide not to register to the IRIM3D event, but want to participate to the workshop;
    % \item to the organisation committee members who decide not to register to the IRIM3D event, but want to participate to the workshop;
    % \item to the authors of accepted manuscript who decide not to register to the IRIM3D event, but want to participate to the workshop.
\end{enumerate}



\end{document}
